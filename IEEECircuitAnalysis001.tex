\documentclass[conference]{IEEEtran}

\usepackage{cite}
\usepackage{amsmath,amssymb,amsfonts}
\usepackage{algorithmic}
\usepackage{graphicx}
\usepackage{textcomp}
\usepackage{xcolor}

\def\BibTeX{{\rm B\kern-.05em{\sc i\kern-.025em b}\kern-.08em
    T\kern-.1667em\lower.7ex\hbox{E}\kern-.125emX}}
    
\begin{document}

\title{Large Signal Analysis and architecture
patterns in analog design.}

\thanks{Thanks to support from IndiaRXiv and OSF.}


\author{\IEEEauthorblockN{1\textsuperscript{st} Dr Anil Kumar Bheemaiah}
\IEEEauthorblockA{\textit{Lead Researcher and Owner, Bheemaiah Corporation} \\
\textit{Bheemaiah Corporation, Seattle , USA}\\
\\
miyawaki@yopmail.com}

}




\maketitle

\begin{abstract}
Design patterns inspired from the field of architecture, have a strong presence in software engineering,
with the leading contribution by the gang of four, circuit design lacks formal architectural definitions and
design patterns, definitions, although, BEAM or MFA I design has a modular design structure with design
patterns. In this paper we introduce a Tensor definition for architectural patterns, with a stability analysis
of large signal behaviour, leading to definitions of associative memory, state diagrams and digital
architectural paradigms in analog design. In future work, we design natural language like grammars to
tensor representations in a formal complexity theory of Lie Computability and monotonic functions.


\end{abstract}

\begin{IEEEkeywords}
design patterns, structural, functional, behavioural, architecture, analog design, design thinking, constructivism, BEAM robotics, gestalt.
\end{IEEEkeywords}

\section{Introduction}
\subsection{What:}
Structural and Functional architectural patterns in analog design, with small signal and large signal
analysis in analog design, illustrated with Nv neuron architectures.
\subsection{How:}
Reconfigurable design blocks for op amps and BJT/FETs are explored with reconfigurable control buses,
and complex impedances, a combination of capacitive, inductive and resistive elements, all amenable to
low scale integrated design as FlexICs. With this modular core for use in reconfigurable BEAM robotics,
as an illustration of the use of design patterns.
\subsubsection{The Introduction}
Opamp and transistor based circuits are
designed with patterns using control loops
and noise engineering, based on feedback
control systems and many architectural
patterns, behavioural, structural and
functional. We describe several such
patterns and illustrate examples with model
and behaviour driven design, in Nv neuron
based Tensor topologies.
Small signal models of two port networks
are translated into simple model based
approximations of active impedances. In
impedance transfer logic, the atomicity of
analog design, we present a large signal
analysis in tensor definitions of complexity,
with a dynamical systems definition of two
ports with noise engineering, for S/N ratios
and gain bandwidth tradeoff in design
architectures.


\section{Background}
In the previous publications, we have
defined a tensor architecture for Nv and Nu
neurons for MFA I architectures, in this
paper we generalize design patterns for
architecture for fulfilment of tensor
topologies.

\section{Architectural Patterns.}

\subsection{Current Mirror Pattern.}
(“Chapter 11: The Current Mirror [Analog
Devices Wiki]” n.d.)
\subsection{Patterns with a reconfigurable opamp
(ROP).}

\begin{itemize}
\item denoted in CaC is an opamp(in+,in-, out),
\item impedence_mux_i,  i = 1 to 4 ,
\item impedence_mux_opamp+(in, out),
\item impedence_mux_opamp-(in,out),
\item imepdence_feedback+(in,out) and
\item impedence_feedback-(in,out).
\end{itemize}

The circuit graph is straightforward from
these elements with the impedance being a
complex value with a multiplexed hard
wired control bus, [in+,in-,fb+,fb-], with
various values for various circuit topologies.
\subsection{Opamp Gain, Negative Feedback.}
An ROP with [in+,in-,fb+,fb-] = [1, 0, 1,0]
\subsection{Active Impedance, Negative Feedback.}
ROP with [in+,in-,fb+,fb-] = [1, 1, 1,0]
\subsection{Opamp, Differential Amplifier.}
ROP with [in+,in-,fb+,fb-] = [1, 1, 0,0]
\subsection{Active Impedance, Positive Feedback}
ROP with [in+,in-,fb+,fb-] = [1, 1, 0, 1]


\section{Transistor based patterns -  both
BJT and FET  (RTB)}

\begin{itemize}
\item transistor(base,in , out), 
\item impedence_mux_i, i = 1 to 4 ,
\item impedence_base_1(in, out),
\item impedence_base_2(in,out),
\item imepdence_in( in,out) and
\item impedence_out(in,out).
\end{itemize}

\sunsection{}
The RTB allows for a biased reconfigurable
transistor, allowing for many topologies on a
control vector [b1, b2, in,out] and 4 port [b1,
b2, in, out] for creation of Nv, Nu and n-core
biomorphs as an illustration.

\section{Functional and structural patterns.}
\begin{enumerate}
\item  Frequency Modulation.(functional) with control [1, 1, 1, 0] or [1, 1, 1, 1] with an impedanceIn value for an oscillator.
\item  Amplitude Modulation (functional) with control [1, 1, 1, 1] and impedanceOut with capacitive and resistive values.
\item Pulse code modulation.(functional) Bicore, multivibrator with variable duty cycle, two [1,1,1,1] with positive feedback.
\item Central pattern generator, many bicore designs, with positive feedback for noise engineering.
\item Master-slave pattern.: Bicore flip-flop designs, with a feedback control loop for master - slave following, with capacitive impedance.
\item  Handshake Protocol Pattern.: multicore, state machine based design, with atomic bicore state unit. State machine for hand shake.
\item Shared Memory Pattern: Hysteresis based memory or limit cycles based bicore designs, for complex dynamical patterns.
\item  I/O interrupt - asynchronous pattern.: State machine based multicore designs with interrupt controls.
\item  Associative memory pattern.: bicore designs as dynamical systems with internal states and adaptive resonance theory for associative memory.(Contributors to Wikimedia projects 2016)
\item  Hysteresis Pattern. (Contributors to Wikimedia projects 2003) Bicore with positive feedback.
\item  State Machines. Bicore hysteresis forms a simple state machine, with a tandem n-core, creates a state machine. Each non schmitt trigger machine, with positive feedback has a rich state , phase diagram, with limit cycles for positive states and transitions from one state to another.
\end{enumerate}

\section {Formal definitions in Large signal
analysis.}
Large signal analysis of two port networks
leads to non linear active impedances and
tensor definitions for impedance transfer in
opamp or transistor based designs, Stability
theory in dynamical systems is well
described in literature.(Corinto, Lanza, and
Gilli 2007)(Bheemaiah, n.d.)
Dc2Dc converters present a rich gamut of
behaviours in stability analysis, leading to
state machines with limit cycles and
bifurcations of stable states and transitions
ranging from ergodicity to bifurcations to
persistence effects, in states.(“[No Title]”
n.d.)(“[No Title]” n.d.)



\section{Discussion.}
Formal definitions of state systems in the
context of stability theory, protocols for
handshakes and maps, from lyapunov
functions as the large signal solutions to
field equations in two port networks for
active impedance and impedance transfer
have been described in this paper.
The Tensor formulation , described as a
generalized [T \(theta\)], is implemetable directly in
state machine definitions and maps. Formal
theory of analog architectures is defined in
monotonic functions and asymptotics in
complexity definitions of convergence and
stability. This is defined in a future paper on
Lie computability and Lie lattice theories of
circuit analysis.









\begin{thebibliography}{00}
\bibitem{b1}Bheemaiah, Anil Kumar. n.d. “BEAM Autopilot.” https://doi.org/10.35543/osf.io/st2xa.
\bibitem{b2}“Chapter 11: The Current Mirror [Analog Devices Wiki].” n.d. Accessed June 15, 2020. https://wiki.analog.com/university/courses/electronics/text/chapter-11.
\bibitem{b3}Contributors to Wikimedia projects. 2002. “Design Patterns,” February. https://en.wikipedia.org/wiki/Design_Patterns.
\bibitem{b4}———. 2003. “Schmitt Trigger,” November. https://en.wikipedia.org/wiki/Schmitt_trigger.
\bibitem{b5}———. 2016. “Fusion Adaptive Resonance Theory,” March. 	https://en.wikipedia.org/wiki/Fusion_adaptive_resonance_theory.
\bibitem{b6}Corinto, Fernando, Valentina Lanza, and Marco Gilli. 2007. “Limit Cycles and Bifurcations in Nonlinear Oscillatory Networks.” 2007 IEEE International Symposium on Circuits and Systems. https://doi.org/10.1109/iscas.2007.378102.
\bibitem{b7}“[No Title].” n.d. Accessed June 15, 2020a. https://www.worldscientific.com/doi/abs/10.1142/S0218127416501662.
\bibitem{b8}———. n.d. Accessed June 15, 2020b. https://www.mdpi.com/1996-1073/11/10/2747/pdf.
\end{thebibliography}



\end{document}
